\documentclass{article}

\title{My first \LaTeX \ document}
\author{Aseem Mahajan}
\date{September 5, 2023}

\begin{document}
\maketitle

%Paragraph 1%
Statistical concepts and methods are not only useful but indeed often indispensable in understanding the world around us. They provide ways of gaining new insights into the behaviour of many phenomena that you will encounter in your chosen field of specialization in engineering or science.
%Paragraph ends%

%Paragraph 2%
The discipline of statistics teaches us how to make intelligent judgements and informed decisions in the presence of uncertainty and variation. Without uncertainty or variation, there would be little need for statistical methods or statisticians. If every component of a particular type had exactly the same lifetime, if all resistors produced by a certain manufacturer had the same resistance value, if pH determinations for soil specimens from a particular locale give identical results, and so on, then a single observationn would reveal all desired information.
%Paragraph ends%

%Paragraph 3%
An interesting manifestation of variation arises in the course of performing emissions testing on motor vehicles. The expense and time requirements of the Federal Test Procedure (FTP) preclude its widespread use in vehicle inspection programs. As a result, many agencies have developed less costly and quicker tests, which it is hoped replicate FTP results. According to the journal article "Motor Vehicle Emissions Variability" \emph{J. of the Air and Waste Mgmt. Assoc., 1996: 667-675}, the acceptance of the FTP as a gold standard has led to the widespread belief that repeated measurements in the same vehicle would yield identical (or nearly identical) results. The authors of the article applied the FTP to seven vehicles characterized as "high emitters." Here are the results for one such vehicle:
%Paragraoh ends%
%Example table 1%
\begin{center}\[
    \begin{array}{ccccc}
         HC \ (gm/mile) & 13.8 & 18.3 & 32.2 & 32.5\\
         CO \ (gm/mile) & 118 & 149 & 232 & 236
    \end{array}\]
\end{center}

%Paragraph 4%
The substantial variation in both the HC and CO measurements casts considerable doubt on conventional wisdom and makes it much more difficult to make precise assessments about emission levels.
%Paragraph ends%

%Paragraph 5%
How can statistical techniques be used to gather information and draw conclusions? Suppose, for example, that a materials engineer has developed a coating for retarding corrosion in metal pipe under specified circumstances. If this coating is applied to diffrent segments of pipe , variation in environmental conditions and in the segment themselves will result in more susbstancial corrosion on some segments thamn on others. Method of statistical analysis could be used on data from such an experiment to decide whether the average amount of corrosion exceeds an upper specification limit of some sort or to predict how much corrosion will occur on a single piece of paper. 
%Paragraph ends%
\end{document}
